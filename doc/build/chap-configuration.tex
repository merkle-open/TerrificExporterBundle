\hsection{Configuration}

\newcommand{\optiondesc}[2]{
	\noindent
	\begin{minipage}{\textwidth}
	\vspace{1em}
	\begin{adjustwidth}{0cm}{0cm}
		\noindent \textbf{#1}\\
		\vspace{-1em}
		\begin{adjustwidth}{\parindent}{0cm}
			#2
		\end{adjustwidth}
	\end{adjustwidth}
	\end{minipage}
}

All configuration settings goes beyond a terrific\_exporter node within your environment configuration file.\\
\\
% ################################################################################################################################
\optiondesc{build\_local\_paths: (true/false)}{
	If build\_local\_paths is enabled the exporter will change all urls within html and css files to match within the exported package.
	\\See actions '\hyperlink{sec-Actions-ExportAssets}{ExportAssets}' and '\hyperlink{sec-Actions-ExportViews}{ExportViews}'.\\
}

% ################################################################################################################################
\optiondesc{build\_js\_doc: (true/false)}{
	Enables the export of a javascript documentation. The documentation is generated using YUIDoc.
	\\See action '\hyperlink{sec-Actions-BuildJSDoc}{BuildJSDoc}'.\\
}

% ################################################################################################################################
\optiondesc{build\_settings: (path)}{
	This setting should target to a build.ini file. Within this file there are only settings for the projektname an versioning data.
	\\See section '\hyperlink{sec-build.ini}{build.ini}'.\\
}

% ################################################################################################################################
\optiondesc{build\_path: (path)}{
	Has to target to a path which is the export target path.\\
}

% ################################################################################################################################
\optiondesc{export\_with\_version: (true/false)}{
	Set to true if the exporter should build zips/folders with version numbers within its name.
	\\See section '\hyperlink{sec-build.ini}{build.ini}'.\\
}

% ################################################################################################################################
\optiondesc{autoincrement\_build: (true/false)}{
	True if the exporter should increase the revision after each build.
	\\See section '\hyperlink{sec-build.ini}{build.ini}'.\\
}

% ################################################################################################################################
\optiondesc{validate\_js: (true/false)}{
	Activates the validation of javascript. Validation is done using jshint.
	\\See action '\hyperlink{sec-Actions-ValidateJS}{ValidateJS}'.\\
}

% ################################################################################################################################
\optiondesc{validate\_css: (true/false)}{
	Activate the validation of css. Validation is done using csslint.
	\\See action '\hyperlink{sec-Actions-ValidateCSS}{ValidateCSS}'.\\
}

% ################################################################################################################################
\optiondesc{optimize\_image: (true/false)}{
	Set to true to optimize images in the output directory. Optimization is done using jpegoptim, optipng and advpng.
	\\See action '\hyperlink{sec-Actions-OptimizeImages}{OptimizeImages}'.\\
}

% ################################################################################################################################
\optiondesc{export\_views: (true/false)}{
	Activates the export of the views marked with a @Export annotation.
	\\See action '\hyperlink{sec-Actions-ExportViews}{ExportViews}'.\\
}

% ################################################################################################################################
\optiondesc{export\_modules: (true/false)}{
	Activates the export of plain module html. The url within this modules are not rewriten even if the build\_local\_paths option is activated.
	\\See action '\hyperlink{sec-Actions-ExportModules}{ExportModules}'.\\
}

% ################################################################################################################################
\optiondesc{export\_type: (string: folder/zip)}{
	Set the export type if the export should be done as folder or as a zip.\\
}

% ################################################################################################################################
\optiondesc{build\_actions: (list of objects)}{
	These option allows to setup a build chain. Here you can append project related exporting tasks or change the builtin order.
	\begin{itemize}
	      \item{Terrific\textnormal{\textbackslash}ExporterBundle\textnormal{\textbackslash}Actions\textnormal{\textbackslash}ClearAction}
	      \item{Terrific\textnormal{\textbackslash}ExporterBundle\textnormal{\textbackslash}Actions\textnormal{\textbackslash}BuildJSDoc}
	      \item{Terrific\textnormal{\textbackslash}ExporterBundle\textnormal{\textbackslash}Actions\textnormal{\textbackslash}ValidateJS}
	      \item{Terrific\textnormal{\textbackslash}ExporterBundle\textnormal{\textbackslash}Actions\textnormal{\textbackslash}ValidateCSS}
	      \item{Terrific\textnormal{\textbackslash}ExporterBundle\textnormal{\textbackslash}Actions\textnormal{\textbackslash}ValidateModules}
	      \item{Terrific\textnormal{\textbackslash}ExporterBundle\textnormal{\textbackslash}Actions\textnormal{\textbackslash}ValidateViews}
	      \item{Terrific\textnormal{\textbackslash}ExporterBundle\textnormal{\textbackslash}Actions\textnormal{\textbackslash}GenerateSprites}
	      \item{Terrific\textnormal{\textbackslash}ExporterBundle\textnormal{\textbackslash}Actions\textnormal{\textbackslash}ExportImages}
	      \item{Terrific\textnormal{\textbackslash}ExporterBundle\textnormal{\textbackslash}Actions\textnormal{\textbackslash}ExportAssets}
	      \item{Terrific\textnormal{\textbackslash}ExporterBundle\textnormal{\textbackslash}Actions\textnormal{\textbackslash}OptimizeImages}
	      \item{Terrific\textnormal{\textbackslash}ExporterBundle\textnormal{\textbackslash}Actions\textnormal{\textbackslash}ExportModules}
	      \item{Terrific\textnormal{\textbackslash}ExporterBundle\textnormal{\textbackslash}Actions\textnormal{\textbackslash}ExportViews}
	      \item{Terrific\textnormal{\textbackslash}ExporterBundle\textnormal{\textbackslash}Actions\textnormal{\textbackslash}ExportChangelogs}
	\end{itemize}
	See section '\hyperlink{chap-Actions}{Actions}' or '\hyperlink{chap-Extending}{Extending}' for further information.\\
}

% ################################################################################################################################
\optiondesc{pathtemplates: (set of string values)}{
	The pathtemplates option allows you to customize you paths within your export package. All paths begin with a starting '/' each given directory will begin relative to the given export\_path. So a value like '/img/common' will end up in '/exportpath/img/common'. The optional \mbox{\%module\%} variable within will be resolved by the PathResolver into a modulename. This variable only get matched in 'module\_*' options.
	\\
	It is possible to set paths for the following types of files:
	\begin{itemize}
		\item{image: (default: '/img/common')}
		\item{font: (default: '/fonts')}
		\item{css: (default: '/css')}
		\item{js: (default: '/js')}
		\item{view: (default: '/views')}
		\item{changelog: (default: '/changelogs')}
		\item{diff: (default: '/changelogs/diff')}

		\item{module\_image: (default: '/img/\%module\%')}
		\item{module\_font: (default: '/fonts/\%module\%')}
		\item{module\_css: (default: '/css/\%module\%')}
		\item{module\_js: (default: '/js/\%module\%')}
		\item{module\_view: (default: '/views/\%module\%')}
	\end{itemize}

	\noindent See actions '\hyperlink{sec-Actions-ExportModules}{ExportModules}', '\hyperlink{sec-Actions-ExportAssets}{ExportAssets}' and '\hyperlink{sec-Actions-ExportViews}{ExportViews}'.\\
}

% ################################################################################################################################
\optiondesc{sprites: (list of objects)}{
	Here you can setup sprite information. The exporter will build the sprites with the given data.
	\\See section '\hyperlink{sec-Actions-GenerateSprites}{GenerateSprites}'.\\
}


% ################################################################################################################################
\optiondesc{changelog\_path: (directory)}{
	Set this value to a valid path which contains the changelogs for your project. If this folder doesn't exist no changelogs are appended. The value is also relative from the projectfolder like \mbox{'build\_path'} or \mbox{'build\_settings'}. Defaultvalue \mbox{'build/changelogs'}
}


% ################################################################################################################################
\newpage
\subsection{Example configuration}

\begin{verbatim}
terrific_exporter:
	build_local_paths:        true
	build_js_doc:             true
	build_settings:           "build/build.ini"
	build_path:               "build/"
	export_with_version:      false
	autoincrement_build:      true
	validate_js:              false
	validate_css:             false
	validate_html:            false
	optimize_images:          true
	export_views:             true
	export_modules:           true
	export_type:              folder

	pathtemplates:
        image:     "/bilder/common23"
        font:      "/schriften"
        css:       "/styles"
        js:        "/scripts"
        view:      "/html"
        changelog: "/changelogs"
        diff:	   "/changelogs/diff"

        module_image: "/module/%%module%%/bilder"
        module_font:  "/module/%%module%%/schriften"
        module_css:   "/module/%%module%%/styles"
        module_js:    "/module/%%module%%/scripts"
        module_view:  "/module/%%module%%/html"

	sprites:
              - {
                directory: "PROD/internet_sprite_icons",
                target: "web/img/sprite_icons.png", item: { height: 50, width: 100 }
              }

\end{verbatim}

% ################################################################################################################################
\subsection{YUIDoc}
The exporter will use the yuidoc configuration file named yuidoc.json within the app/config directory. The syntax of the file could be read on the YUIDoc page \url{http://yui.github.com/yuidoc/args/index.html}.

% ################################################################################################################################
\subsection{csslint}
Csslint will use a configuration if one is found under app/config, if there is no configuration named csslint.cfg the exporter will use its default configuration found within\\
\%kernel.root\_dir\%/vendor/senuphtyz/terrific-exporter-bundle/Terrific/ExporterBundle/Resources/config. \\
\\
If you want to specify a different csslint.cfg you should take a copy from the default and change the settings within. Show a list of available options just enter the following command in your console.

\begin{verbatim}
csslint --list-rules
\end{verbatim}

\subsection{jshint}
JShint will also use a configuration file named jshint.json, normally found within app/config. If no configuration file is found there the default within ....ExporterBundle/Resources/config is used.\\
\\
See \url{http://www.jshint.com/docs/} configuration settings.\\

% ################################################################################################################################
\hsubsection{build.ini}

\begin{verbatim}
[version]
name=Terrific
major=0
minor=0
build=0
\end{verbatim}
\noindent Normally the default build.ini will look like this. If you specify a build.ini with option 'build\_settings' that is not available, the exporter will create it from the default. Depending on option 'autoincrement\_build' the build number is incremented each export.

