\section{Configuration}

All configuration goes beyond a terrific\_exporter node within the config\_export.yml.\\
\\
\textbf{build\_local\_paths: (true/false)}\\
\indent If build\_local\_paths is enabled the exporter will change all urls within html and css files to match within the exported package.\\
\\
\textbf{build\_js\_doc: (true/false)} \\
\indent Enables the export of a javascript documentation. The documentation is generated using YUIDoc. \\
\\
\textbf{build\_settings: (path)}\\
\indent This setting should target to a build.ini file. Within this file there are only settings for the \\projektname an versioning data.\\
\\
\textbf{build\_path: (path)}\\
\indent Has to target to a path which is the export target path.\\
\\
\textbf{export\_with\_version: (true/false)}\\
\indent Set to true if the exporter should build zips/folders with version numbers within its name.\\
\\
\textbf{autoincrement\_build: (true/false)}\\
\indent True if the exporter should increase the revision after each build.\\
\\
\textbf{validate\_js: (true/false)}\\
\indent Activates the validation of javascript. Validation is done using jshint.\\
\\
\textbf{validate\_css: (true/false)}\\
\indent Activate the validation of css. Validation is done using csshint.\\
\\
\textbf{optimize\_image: (true/false)}\\
\indent Set to true to optimize images in the output directory. Optimization is done using jpegoptim, \\optipng and advpng.\\
\\
\textbf{export\_views: (true/false)}\\
\indent Activates the export of the views marked with a @Export annotation.\\
\\
\textbf{export\_modules: (true/false)}\\
\indent Activates the export of plain module html. The url within this modules are not rewriten even if \\the build\_local\_paths option is activated.\\
\\
\textbf{export\_type: (string: folder/zip)}\\
\indent Set the export type if the export should be done as folder or as a zip.\\
\\
\textbf{build\_actions: (list of objects)}\\
\indent These option allows to setup a build chain. Here you can append project related exporting tasks or change the buildin order.\\
\\
\textbf{sprites: (list of objects)}\\
\indent Here you can setup sprite information. The exporter will build the sprites with the given data.\\
\\


\subsection{Example configuration}

\begin{verbatim}

terrific_exporter:
	build_local_paths:        true
	build_js_doc:             true
	build_settings:           "build/build.ini"
	build_path:               "build/"
	export_with_version:      false
	autoincrement_build:      true
	validate_js:              false
	validate_css:             false
	validate_html:            false
	optimize_images:          true
	export_views:             true
	export_modules:           true
	export_type:              folder

	build_actions:
	      - Terrific\ExporterBundle\Actions\ClearAction
	      - Terrific\ExporterBundle\Actions\BuildJSDoc
	      - Terrific\ExporterBundle\Actions\ValidateJS
	      - Terrific\ExporterBundle\Actions\ValidateCSS
	      - Terrific\ExporterBundle\Actions\ValidateModules
	      - Terrific\ExporterBundle\Actions\ValidateViews
	      - Terrific\ExporterBundle\Actions\GenerateSprites
	      - Terrific\ExporterBundle\Actions\ExportImages
	      - Terrific\ExporterBundle\Actions\ExportAssets
	      - Terrific\ExporterBundle\Actions\OptimizeImages
	      - Terrific\ExporterBundle\Actions\ExportModules
	      - Terrific\ExporterBundle\Actions\ExportViews

	sprites:
	      - { directory: "PROD/internet_sprite_icons", target: "web/img/sprite_icons.png", item: { height: 50, width: 100 }}

\end{verbatim}

\subsection{YUIDoc}
The exporter will use the yuidoc configuration file named yuidoc.json within the app/config directory. The syntax of the file could be read on the YUIDoc page http://yui.github.com/yuidoc/args/index.html. 

\subsection{csslint}
Csslint will use a configuration if one is found under app/config, if there is no configuration named csslint.cfg the exporter will use its default configuration found within \\ 
\%kernel.root_dir\%/vendor/senuphtyz/terrific-exporter-bundle/Terrific/ExporterBundle/Resources/config. \\

\subsection{jshint}
JShint will also use a configuration file named jshint.json, normally found within app/config. If no configuration file is found there the default within ....ExporterBundle/Resources/config is used.\\
\\
See http://www.jshint.com/docs/ for option explanation.\\

\subsection{build.ini}

\begin{verbatim}
[version]
name=Terrific
major=0
minor=0
build=0
\end{verbatim}
\noindent Normally the default build.ini will look like this. 