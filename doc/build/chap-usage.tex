\newcommand{\cmdoptiondesc}[2]{
    \noindent \textbf{#1}\\
    \vspace{-1em}
    \begin{adjustwidth}{\parindent}{0cm}
        #2
    \end{adjustwidth}
    \vspace{1em}
}


\hsection{Usage}

To startup an export use the following command.

\begin{verbatim}
    app/console build:export --env=export --no-debug
\end{verbatim}



\subsection{Commandline options}

\cmdoptiondesc{--no-image-optimization}{
    Overrides configuration setting optimize\_images and starts the chain without optimizing Images. Skip actions:
    \begin{itemize}
          \item{Terrific\textnormal{\textbackslash}ExporterBundle\textnormal{\textbackslash}Actions\textnormal{\textbackslash}OptimizeImages}
    \end{itemize}
}

\cmdoptiondesc{--no-js-doc} {
    Do not build a javascript documentation for this export run. This does skip the following actions:
    \begin{itemize}
        \item{Terrific\textnormal{\textbackslash}ExporterBundle\textnormal{\textbackslash}Actions\textnormal{\textbackslash}BuildJSDoc}
    \end{itemize}
}

\cmdoptiondesc{--no-validation} {
    Ignores validation configuration and skip the following actions:
    \begin{itemize}
          \item{Terrific\textnormal{\textbackslash}ExporterBundle\textnormal{\textbackslash}Actions\textnormal{\textbackslash}ValidateJS}
          \item{Terrific\textnormal{\textbackslash}ExporterBundle\textnormal{\textbackslash}Actions\textnormal{\textbackslash}ValidateCSS}
          \item{Terrific\textnormal{\textbackslash}ExporterBundle\textnormal{\textbackslash}Actions\textnormal{\textbackslash}ValidateModules}
          \item{Terrific\textnormal{\textbackslash}ExporterBundle\textnormal{\textbackslash}Actions\textnormal{\textbackslash}ValidateViews}
    \end{itemize}
}


\subsection{Annotations}

\noindent
\begin{minipage}{\textwidth}
\textbf{@Export}\\
\vspace{-1em}
\begin{adjustwidth}{\parindent}{0cm}
To export a view it is necessary to annotate a controller method with this annotation. It is possible to control a set of options for each view directly within this annotation. This example is only valid if you don't have to export a localized version of a view. If no environment is given this view will exported in all environments.

\begin{verbatim}
 @Export(
     name="viewname.html",
     environment="env1,env2,..."
 )
\end{verbatim}
\end{adjustwidth}
\vspace{1em}
\end{minipage}

\noindent
\begin{minipage}{\textwidth}
\textbf{@LocaleExport}\\
\vspace{-1em}
\begin{adjustwidth}{\parindent}{0cm}
Each view must have their own for each locale which should exported. To set different settings for each locale you have to use the @LocaleExport annotation. This annotation is used in combination with the @Export annotation. Setting up locale exportation will disable exporting the default language which means you have to annotate \textbf{all} locales that should exported. If no environment is given this view will exported in all environments. \\
\\
Usage with @Export:
\begin{verbatim}
@Export(
    @LocaleExport(name="viewname_de.html", locale="de"),
    @LocaleExport(name="viewname_en.html", locale="en", environment="env1"),
    @LocaleExport(name="viewname_lv.html", locale="lv", environment="env2"),
    ....
)
\end{verbatim}
\end{adjustwidth}
\vspace{1em}
\end{minipage}

